%% abtex2-modelo-trabalho-academico.tex, v-1.9.6 laurocesar
%% Copyright 2012-2016 by abnTeX2 group at http://www.abntex.net.br/ 
%%
%% This work may be distributed and/or modified under the
%% conditions of the LaTeX Project Public License, either version 1.3
%% of this license or (at your option) any later version.
%% The latest version of this license is in
%%   http://www.latex-project.org/lppl.txt
%% and version 1.3 or later is part of all distributions of LaTeX
%% version 2005/12/01 or later.
%%
%% This work has the L PPL maintenance status `maintained'.
%% 
%% The Current Maintainer of this work is the abnTeX2 team, led
%% by Lauro César Araujo. Further information are available on 
%% http://www.abntex.net.br/
%%
%% This work consists of the files abntex2-modelo-trabalho-academico.tex,
%% abntex2-modelo-include-comandos and abntex2-modelo-references.bib
% ------------------------------------------------------------------------
% ------------------------------------------------------------------------
% abnTeX2: Modelo de Trabalho Academico (tese de doutorado, dissertacao de
% mestrado e trabalhos monograficos em geral) em conformidade com 
% ABNT NBR 14724:2011: Informacao e documentacao - Trabalhos academicos -
% Apresentacao
% ------------------------------------------------------------------------
% ------------------------------------------------------------------------
% Personalização para o modelo Udesc 2020 7. ed. revisada e modificada
% MANUAL_2020_09_07_1599489825065_12510.pdf
% Autor: Felipe Joel Zimann (felipezimann@hotmail.com)
% Data: 02/12/2020 v1.0
% Data: 13/02/2021 v1.0.1 alterado tamanho numeração da página para 10pt
% ------------------------------------------------------------------------
% ------------------------------------------------------------------------

\documentclass[
	12pt,					% tamanho da fonte
	openright,				% capítulos começam em pág ímpar (insere página vazia caso preciso)
	oneside,				% para impressão em recto e verso (twoside). Oposto a (oneside)
	a4paper,				% tamanho do papel. 
	chapter=TITLE,			% títulos de capítulos convertidos em letras maiúsculas
	section=TITLE,			% títulos de seções convertidos em letras maiúsculas
	sumario=abnt-6027-2012,
	english,				% idioma adicional para hifenização
	brazil,					% o último idioma é o principal do documento
	fleqn,					% equações alinhadas a esquerda (UDESC/CCT)+
	]{abntex2}

% ----------------------------------------------------------
% Pacotes básicos 
% ----------------------------------------------------------
\usepackage{amsmath}							% Pacote matemático
\usepackage{amssymb}							% Pacote matemático
\usepackage{amsfonts}							% Pacote matemático
%\usepackage{lmodern}							% Usa a fonte Latin Modern		
\usepackage{mathptmx} 							% Usa a fonte Times New Roman	 (UDESC/CCT)
\usepackage[T1]{fontenc}						% Selecao de codigos de fonte.
\usepackage[utf8]{inputenc}						% Codificacao do documento (conversão automática dos acentos)
\usepackage{lastpage}							% Usado pela Ficha catalográfica
\usepackage{indentfirst}						% Indenta o primeiro parágrafo de cada seção.
\usepackage[dvipsnames,table]{xcolor}			% Controle das cores
\usepackage{graphicx}							% Inclusão de gráficos
\usepackage{microtype} 							% para melhorias de justificação
\usepackage{lipsum}								% para geração de dummy text
\usepackage[brazilian,hyperpageref]{backref}	% Paginas com as citações na bibl
\usepackage[alf,abnt-emphasize=bf,abnt-full-initials=yes]{abntex2cite}					% Citações padrão ABNT
%\usepackage[num]{abntex2cite}					% Citações padrão ABNT numérica
\usepackage{adjustbox}							% Pacote de ajuste de boxes
\usepackage{subcaption}							% Inclusão de Subfiguras e sublegendas		
\usepackage{enumitem}							% Personalização de listas
\usepackage{siunitx}							% Grandezas e unidades
\usepackage[section]{placeins}					% Manter as figuras delimitadas na respectiva seção com a opção [section]
\usepackage{multirow}							% Multi colunas nas tabelas
\usepackage{array,tabularx} 					% Pacotes de tabelas
\usepackage{booktabs}							% Pacote de tabela profissonal
\usepackage{rotating}							% Rotacionar figuras e tabelas
\usepackage{xfrac}								% Fazer frações n/d em linha
\usepackage{bm}									% Negrito em modo matemático
\usepackage{xstring}							% Manipulação de strings
\usepackage{pgfplots}							% Pacote de Gráficos
\usepackage{tikz}								% Pacote de Figuras
\usepackage[american, cuteinductors,smartlabels, fulldiode, siunitx, americanvoltages, oldvoltagedirection, smartlabels]{circuitikz}						% Pacote de circuitos elétricos
\usepackage{chemformula}						% Pacote para fórmulas químicas
\usepackage{chngcntr}							% Pacte usado para deixar numeração de equações sequencial (UDESC/CCT)
\counterwithout{equation}{chapter}
% fonte: https://latex.org/forum/viewtopic.php?t=15392

% Comando para deixar numeração das equações contínua (1), (2), (3)... ao invés de organizar por capítulos (1.1)(1.2)... (2.1)(2.2)
%\renewcommand{\theequation}{\arabic{equation}}

%\numberwithin{equation}{section}


% Cabecalho cabeçalho somente com numeração de página 10pt
\makepagestyle{PagNumReduzida}
\makeevenhead{PagNumReduzida}{\ABNTEXfontereduzida\thepage}{}{}
\makeoddhead{PagNumReduzida}{}{}{\ABNTEXfontereduzida\thepage}
%fonte: https://github.com/abntex/abntex2/wiki/HowToCustomizarCabecalhoRodape
%fonte: Manual memoir seção 7.3 pg. 111 pdf http://linorg.usp.br/CTAN/macros/latex/contrib/memoir/memman.pdf 

% Personalização das opções das listas
\setlist[itemize]{leftmargin=\parindent}

% Citação online --- MODIFICAR ---
\newcommand{\citeshort}[1]{\citeauthoronline{#1}~(\citeyear{#1})}

\newcommand{\me}[1]{Elaborado pelo autor (#1).}

% Configuração do pgfplots
\pgfplotsset{compat=newest} %compat=1.14
\pgfplotsset{plot coordinates/math parser=false} 
\newlength\figureheight 
\newlength\figurewidth 

% Libraries do TiKz
\usetikzlibrary{quotes,angles,arrows}
\usetikzlibrary{through,calc,math}
\usetikzlibrary{graphs,backgrounds,fit}
\usetikzlibrary{shapes,positioning,patterns,shadows}
\usetikzlibrary{decorations.pathreplacing}
\usetikzlibrary{shapes.geometric}
\usetikzlibrary{arrows.meta}
\usetikzlibrary{external}

%\tikzexternalize[]
%\tikzexternalenable
%\tikzexternalize
%\tikzexternaldisable
%\tikzset{external/force remake}
%\tikzexternalize[shell escape=-enable-write18]

% Configurações do CircuiTiKz
\ctikzset{bipoles/thickness=1}
%\ctikzset{bipoles/length=1.2cm}
\ctikzset{monopoles/ground/width/.initial=.2}
\ctikzset{bipoles/resistor/height=0.25}
\ctikzset{bipoles/resistor/width=0.6}
\ctikzset{bipoles/capacitor/height=0.5}
\ctikzset{bipoles/capacitor/width=0.15}
\ctikzset{bipoles/generic/height=0.25}
\ctikzset{bipoles/generic/width=0.6}
%\ctikzset{bipoles/capacitor polar/length=0.5}
%\ctikzset{bipoles/diode/height=.375}
%\ctikzset{bipoles/diode/width=.3}
%\ctikzset{tripoles/thyristor/height=.8}
%\ctikzset{tripoles/thyristor/width=1}
\ctikzset{bipoles/vsourcesin/height=.5}
\ctikzset{bipoles/vsourcesin/width=.5}
\ctikzset{bipoles/cvsourceam/height=.6}
\ctikzset{bipoles/cvsourceam/width=.6}
%\ctikzset{tripoles/european controlled voltage source/width=.4}

\tikzstyle{every node}=[font=\footnotesize]
\tikzstyle{every path}=[line width=0.25pt,line cap=round,line join=round]
%\tikzstyle{every path}=[line cap=round,line join=round]


% Definição de cores MATLAB
\definecolor{matlab_blue}{rgb}	{         0,    0.4470,    0.7410}
\definecolor{matlab_orange}{rgb}{    0.8500,    0.3250,    0.0980}
\definecolor{matlab_yellow}{rgb}{    0.9290,    0.6940,    0.1250}
\definecolor{matlab_violet}{rgb}{    0.4940,    0.1840,    0.5560}
\definecolor{matlab_green}{rgb}	{	 0.4660,    0.6740,    0.1880}
\definecolor{matlab_lblue}{rgb}	{    0.3010,    0.7450,    0.9330}
\definecolor{matlab_red}{rgb}	{    0.6350,    0.0780,    0.1840}

% Personalização das legendas
\usepackage[format = plain, %hang
			justification = centering,
			labelsep = endash,
			singlelinecheck = false,
			skip = 6pt,
			listformat = simple]{caption}	

% Personalização das unidades
\sisetup{output-decimal-marker = {,}}
\sisetup{exponent-product = \cdot, output-product = \cdot}
\sisetup{tight-spacing=true}
\sisetup{group-digits = false}

% Personalizações de tipo de colunas de tabelas
\newcolumntype{L}[1]{>{\raggedright\let\newline\\\arraybackslash\hspace{0pt}}m{#1}}
\newcolumntype{C}[1]{>{\centering\let\newline\\\arraybackslash\hspace{0pt}}m{#1}}
\newcolumntype{R}[1]{>{\raggedleft\let\newline\\\arraybackslash\hspace{0pt}}m{#1}}

% Personalizações de cores da UDESC
\definecolor{CapaAmareloUDESC}{RGB}{243,186,83}		% Especializacao
\definecolor{CapaVerdeUDESC}{RGB}{0,112,52}			% Mestrado
\definecolor{CapaVermelhoUDESC}{RGB}{171,35,21}		% Doutorado
\definecolor{CapaAzulUDESC}{RGB}{38,54,118} 		% Pós-Doutorado

% CONFIGURAÇÕES DE PACOTES
% Configurações do pacote backref
% Usado sem a opção hyperpageref de backref
\renewcommand{\backrefpagesname}{Citado na(s) página(s):~}
% Texto padrão antes do número das páginas
\renewcommand{\backref}{}
% Define os textos da citação
\renewcommand*{\backrefalt}[4]{
	\ifcase #1 %
	Nenhuma citação no texto.%
	\or
	Citado na página #2.%
	\else
	Citado #1 vezes nas páginas #2.%
	\fi}%

% alterando o aspecto da cor azul
%\definecolor{blue}{RGB}{41,5,195}

% informações do PDF
\makeatletter
\hypersetup{
	%pagebackref=true,
	pdftitle={\@title}, 
	pdfauthor={\@author},
	pdfsubject={\imprimirpreambulo},
	pdfcreator={LaTeX with abnTeX2},
	pdfkeywords={abnt}{latex}{abntex}{abntex2}{trabalho academico}, 
	colorlinks=true,       		% false: boxed links; true: colored links
	linkcolor=black,          	% color of internal links
	citecolor=black,        	% color of links to bibliography
	filecolor=black,      		% color of file links
	urlcolor=black,
	bookmarksdepth=4
}
\makeatother


\makeatletter
\newcommand{\includetikz}[1]{%
	\tikzsetnextfilename{#1}%
	\input{#1.tex}%
}
\makeatother

% ---
% Possibilita criação de Quadros e Lista de quadros.
% Ver https://github.com/abntex/abntex2/issues/176
%
\newcommand{\quadroname}{Quadro}
\newcommand{\listofquadrosname}{Lista de quadros}

\newfloat[chapter]{quadro}{loq}{\quadroname}
\newlistof{listofquadros}{loq}{\listofquadrosname}
\newlistentry{quadro}{loq}{0}

% configurações para atender às regras da ABNT
\setfloatadjustment{quadro}{\centering}
\counterwithout{quadro}{chapter}
\renewcommand{\cftquadroname}{\quadroname\space} 
\renewcommand*{\cftquadroaftersnum}{\hfill--\hfill}

\setfloatlocations{quadro}{hbtp} % Ver https://github.com/abntex/abntex2/issues/176
% ---


% Espaçamento depois do título
\setlength{\afterchapskip}{0.7\baselineskip}
% O tamanho do parágrafo é dado por:
\setlength{\parindent}{1.25cm}
% Controle do espaçamento entre um parágrafo e outro:
\setlength{\parskip}{0.0cm}  % tente também \onelineskip
%\SingleSpacing % Espaçamento simples 
\OnehalfSpacing % Espaçamento 1,5 (UDESC/CCT)
%\DoubleSpacing	% Espaçamento duplo

% ---
% Margens - NBR 14724/2011 - 5.1 Formato
% ---
\setlrmarginsandblock{3cm}{2cm}{*}
\setulmarginsandblock{3cm}{2cm}{*}
\checkandfixthelayout[fixed]
% ---


% To use externalize consider
%https://tex.stackexchange.com/questions/182783/tikzexternalize-not-compatible-with-miktex-2-9-abntex2-package
%Lauro Cesar digged into the problem until he came with a solution for me to test. And it Works!
%
%According to this link:
%
%The package calc changed the commands \setcounter and friends to be fragile. So you have to make them robust. The example below uses etoolbox with \robustify:
%
\usepackage{etoolbox}
\robustify\setcounter
\robustify\addtocounter
\robustify\setlength
\robustify\addtolength


%% How to silence memoir class warning against the use of caption package?
%% https://tex.stackexchange.com/questions/391993/how-to-silence-memoir-class-warning-against-the-use-of-caption-package
%\usepackage{silence}
%\WarningFilter*{memoir}{You are using the caption package with the memoir class}
%\WarningFilter*{Class memoir Warning}{You are using the caption package with the memoir class}

% --------------------------------------------------------
% INICIO DAS CUSTOMIZACOES PARA A UDESC
% --------------------------------------------------------

% --------------------------------------------------------
% Fontes padroes de part, chapter, section, subsection e subsubsection
% --------------------------------------------------------
% --- Chapter ---
\renewcommand{\ABNTEXchapterfont}{\fontseries{b}} %\bfseries
\renewcommand{\ABNTEXchapterfontsize}{\normalsize}
% --- Part ---
\renewcommand{\ABNTEXpartfont}{\ABNTEXchapterfont}
\renewcommand{\ABNTEXpartfontsize}{\LARGE}
% --- Section ---
\renewcommand{\ABNTEXsectionfont}{\normalfont}
\renewcommand{\ABNTEXsectionfontsize}{\normalsize}
% --- SubSection ---
\renewcommand{\ABNTEXsubsectionfont}{\fontseries{b}} %\bfseries
\renewcommand{\ABNTEXsubsectionfontsize}{\normalsize}
% --- SubSubSection ---
\renewcommand{\ABNTEXsubsubsectionfont}{\itshape}
\renewcommand{\ABNTEXsubsubsectionfontsize}{\normalsize}

\renewcommand{\ABNTEXsubsubsubsectionfont}{\normalfont}
\renewcommand{\ABNTEXsubsubsubsectionfontsize}{\normalsize}
% ---

% --------------------------------------------------------
% Fontes das entradas do sumario
% --------------------------------------------------------

\renewcommand{\cftpartfont}{\ABNTEXpartfont\selectfont}
\renewcommand{\cftpartpagefont}{\normalsize\selectfont}

\renewcommand{\cftchapterfont}{\ABNTEXchapterfont\selectfont}
\renewcommand{\cftchapterpagefont}{\normalsize\selectfont}

\renewcommand{\cftsectionfont}{\ABNTEXsectionfont\selectfont}
\renewcommand{\cftsectionpagefont}{\normalsize\selectfont}

\renewcommand{\cftsubsectionfont}{\ABNTEXsubsectionfont\selectfont}
\renewcommand{\cftsubsectionpagefont}{\normalsize\selectfont}

\renewcommand{\cftsubsubsectionfont}{\normalfont\itshape\selectfont}
\renewcommand{\cftsubsubsectionpagefont}{\normalsize\selectfont}

\renewcommand{\cftparagraphfont}{\normalfont\selectfont}
\renewcommand{\cftparagraphpagefont}{\normalsize\selectfont}

% --------------------------------------------------------
% Usando os pacotes hyperref, uppercase... 
% Para deixar a section do toc uppercase precisa de:
% --------------------------------------------------------
\usepackage{textcase}

\makeatletter

\let\oldcontentsline\contentsline
\def\contentsline#1#2{%
	\expandafter\ifx\csname l@#1\endcsname\l@section
	\expandafter\@firstoftwo
	\else
	\expandafter\@secondoftwo
	\fi
	{%
		\oldcontentsline{#1}{\MakeTextUppercase{#2}}%
	}{%
		\oldcontentsline{#1}{#2}%
	}%
}
\makeatother

% --------------------------------------------------------
% Renomenando as entradas de APÊNDICES E ANEXOS
% --------------------------------------------------------

\renewcommand{\apendicesname}{AP\^ENDICES}
\renewcommand{\anexosname}{ANEXOS}


% Manipulação de Strings
%\RequirePackage{xstring}

% Comando para inverter sobrenome e nome
\newcommand{\invertname}[1]{%
	\StrBehind{#1}{{}}, \StrBefore{#1}{{}}%
}%


% --------------------------------------------------------
% Alterando os estilos de Caption e Fonte
% --------------------------------------------------------
\makeatletter
% Define o comando \fonte que respeita as configurações de caption do memoir ou do caption
\renewcommand{\fonte}[2][\fontename]{%
	\M@gettitle{#2}%
	\memlegendinfo{#2}%
	\par
	\begingroup
	\@parboxrestore
	\if@minipage
	\@setminipage
	\fi
	\ABNTEXfontereduzida
	\configureseparator
	\captiondelim{\ABNTEXcaptionfontedelim}
	\@makecaption{#1}{\ignorespaces #2}\par
	\endgroup}


\captionstyle[\raggedright]{\raggedright}

\makeatother

\setlength{\cftbeforechapterskip}{0pt plus 0pt}
\renewcommand*{\insertchapterspace}{}

\newlength{\mylen}	% New length to use with spacing
\setlength{\mylen}{1pt}

\setlength{\cftbeforechapterskip}{\mylen}
\setlength{\cftbeforesectionskip}{\mylen}
\setlength{\cftbeforesubsectionskip}{\mylen}
\setlength{\cftbeforesubsubsectionskip}{\mylen}
\setlength{\cftbeforesubsubsubsectionskip}{\mylen}


% ---
% Ajuste das listas de abreviaturas e siglas ; e símbolos [Personalizada para UDESC com espaçamento 1,5]
% ---

% ---
% Redefinição da Lista de abreviaturas e siglas [Personalizada para UDESC com espaçamento 1,5]
\renewenvironment{siglas}{%
	\pretextualchapter{\listadesiglasname}
	\begin{symbols} 
		\setlength{\itemsep}{0pt}	% Ajuste para Espaçamento 1,5 (UDESC/CCT)
	}{% 
	\end{symbols}
	\cleardoublepage
}
% ---

% ---
% Redefinição da Lista de símbolos [Personalizada para UDESC com espaçamento 1,5]
\renewenvironment{simbolos}{%
	\pretextualchapter{\listadesimbolosname}
	\begin{symbols}
		\setlength{\itemsep}{0pt}	% Ajuste para Espaçamento 1,5 (UDESC/CCT)
	}{%
	\end{symbols}
	\cleardoublepage
}
% ---





% ---
% FIM DAS CUSTOMIZACOES PARA A  Universidade do Estado de Santa Catarina - UDESC/CCT
% ---





	% Incliu pacotes básicos 

% -----------------------------------------------------------------
% Você pode adicionar seus pacotes a partir desta linha;
% -----------------------------------------------------------------
\usepackage{zed-csp}                            % Pacote Notação Z
\usepackage{datetime}

%\usepackage[showframe,pass]{geometry}
%\usepackage[11,12]{pagesel}

% -----------------------------------------------------------------
% Informações de dados para CAPA e FOLHA DE ROSTO
% -----------------------------------------------------------------
\titulo{Especificação em Notação Z - FarmFlow - Horta Inteligente}%

\autor{Thiago Artur {}Schumann}%
\orientador{Marília Guterres {}Ferreira}%

% ATENÇÃO: O símbolo {} indica o sobrenome para a ficha catalográfica.
% Exemplo: Sherlock Holmes {}da Silva para sobrenomes compostos;
% Exemplo: Arnold Alois {}Schwarzenegger para sobrenome simples.

\instituicao{Universidade do Estado de Santa Catarina, Centro de Educação Superior do Alto Vale do Itajaí, Departamento de Engenharia de Software}%

\tipotrabalho{Trabalho Acadêmico (Graduação)}

\local{Ibirama}%

\date{\the\year}%
% ---

% compila o indice
\makeindex

% -----------------------------------------------------------------
% Início do documento
% -----------------------------------------------------------------
\begin{document}

\selectlanguage{brazil}
\frenchspacing  % Retira espaço extra obsoleto entre as frases.

% -----------------------------------------------------------------
% ELEMENTOS PRÉ-TEXTUAIS
% -----------------------------------------------------------------
\pretextual

% Você pode comentar os elementos que não deseja em seu trabalho;

% A capa pode ser escolhida dentro do arquivo Capa.tex (TCC, Master, Doc, ...)
% ---
% Capa
% ---


% --------------------------------------------------------
% Capa Padrão
% --------------------------------------------------------
\renewcommand{\imprimircapa}{%
	\begin{capa}%
		\center

		{\fontseries{b}\selectfont\MakeTextUppercase{UNIVERSIDADE DO ESTADO DE SANTA CATARINA -- UDESC}}

		{\fontseries{b}\selectfont\MakeTextUppercase{CENTRO DE EDUCAÇÃO SUPERIOR DO ALTO VALE DO ITAJAÍ -- CEAVI  }}

		{\fontseries{b}\selectfont\MakeTextUppercase{ENGENHARIA DE SOFTWARE -- ESO  }}

		\vfill

		{\fontseries{b}\selectfont\MakeTextUppercase{\normalsize\imprimirautor}}

		\vfill
		\begin{center}
		{\fontseries{b}\selectfont\MakeTextUppercase{\imprimirtitulo}}
		\end{center}
		\vfill

		\vfill

		{\fontseries{b}\selectfont\MakeTextUppercase{\imprimirlocal}}
		\par
		{\fontseries{b}\selectfont \imprimirdata}
		\vspace*{1cm}
	\end{capa}
}



\imprimircapa				% Capa padrão

					% Elemento Obrigatório

% -----------------------------------------------------------------
% ELEMENTOS TEXTUAIS
% -----------------------------------------------------------------
\textual

\pagestyle{PagNumReduzida}						% Comando para cabeçalho somente com numeração de página 10pt
\aliaspagestyle{chapter}{PagNumReduzida}		% Deixar numeração da primeira página com tamanho igual ao resto da numeração
% ref.: https://groups.google.com/g/abntex2/c/CP7g8ZMgi-c/m/KjfEnn5b9a4J


% ---- Mantenha está estrutura, assim você deixa o trabalho mais organizado -------

\chapter{Introdução}

	\section{Descrição do sistema}
		O FarmFlow é um projeto que visa tornar o cultivo de plantas mais fácil e eficiente, automatizando as
		atividades. O sistema compreende vários componentes, incluindo sensores de umidade do solo, sensores de
		luminosidade, válvulas de irrigação, controlador de abertura do teto retrátil, microcontroladores (Arduíno) e
		software de controle e integração. A especificação formal do software é uma etapa importante para garantir que
		o sistema funcione corretamente e atenda às necessidades dos usuários.

		A notação Z é uma ferramenta poderosa para a especificação formal de sistemas, pois permite uma descrição
		precisa e completa do comportamento funcional do sistema. Ao usar a notação Z para especificar o software de
		jardim inteligente, é possível garantir que o sistema seja projetado conforme as especificações e requisitos
		definidos pelos usuários, além de ajudar a identificar problemas e manter o sistema ao longo do tempo.



\chapter{Desenvolvimento}
    \section{Requisitos funcionais}
        \begin{table}[!htbp]
            \centering
            \renewcommand{\arraystretch}{1.3}
            \label{tab:quadro_requisitos}
            \begin{tabular}{| L{3cm} | L{11cm} | }
                \hline
                \textbf{RF-01} & O Software deve coletar informações referentes ao sensor de umidade do solo.\\
                \hline

                \hline
                \textbf{RF-02} & O Software deve controlar a cor de um led que representa o estado da umidade do solo.\\
                \hline                   

                \hline
                \textbf{RF-03} & O Software deve controlar a irrigação do solo.\\
                \hline

                \hline
                \textbf{RF-04} & O Software deve coletar informações referentes ao tempo de exposição à luz do solo.\\
                \hline

                \hline
                \textbf{RF-05} & O Software deve controlar o teto retrátil.\\
                \hline

                \hline
                \textbf{RF-06} & O Software deve enviar informações do solo para o Smart Life.\\
                \hline

            \end{tabular}
            \vspace{2mm}
        \end{table}

    \section{Regras de negócio}
        \begin{table}[!htbp]
            \centering
            \renewcommand{\arraystretch}{1.3}
            \label{tab:quadro_regras_negocio}
            \begin{tabular}{| L{3cm} | L{11cm} | }
                \hline
                \textbf{RN-01} & A leitura de umidade referente ao sensor deve realizar leituras a cada 45 minutos.\\
                \hline

                \hline
                \textbf{RN-02} & Quando a leitura de umidade for realizada e a umidade estiver dentro dos limites, luz de umidade fica verde, se não, luz de umidade fica vermelha.\\
                \hline                

                \hline
                \textbf{RN-03} & Quando a leitura da umidade estiver abaixo do nível estipulado, inicie a irrigação, realize leituras a cada segundo e pare a irrigação quando o nível mínimo for atingido. Pois o delay de 1 segundo fará com que o nível de umidade fique acima do nível mínimo.\\
                \hline

                \hline
                \textbf{RN-04} & A leitura de tempo de exposição a luz referente ao sensor deve realizar leituras a cada 45 minutos.\\
                \hline

                \hline
                \textbf{RN-05} & Quando a leitura de luminosidade for realizada e a quantidade de luz já estiver
                insuficiente, o teto retrátil permanece aberto, se não, ele é fechado.\\
                \hline

                \hline
                \textbf{RN-06} & As informações captadas pelos sensores devem ser disponibilizadas para visualização
                no aplicativo Smart Life.\\
                \hline

            \end{tabular}
            \vspace{2mm}
        \end{table}

\newpage

    \section{Especificação formal utilizando notação Z}

        \subsection{Definição de constantes e tipos}

            \begin{zed}
                STATUS ::= LIGADO | DESLIGADO     \\
                COR\_LED ::= VERDE | VERMELHO     \\
                TETO\_RETRATIL ::= ABERTO | FECHADO \\    
                UMIDADE\_MINIMA ::= 60 \\
                UMIDADE\_IDEAL ::= 75 \\
                COR\_LED\_INICIAL :: VERDE \\
                STATUS\_INICIAL :: DESLIGADO \\
                TEMPO\_LUZ\_IDEAL ::= 6 \\
                TETO\_RETRATIL\_INICIAL ::= ABERTO \\
                VAZIO :: = \emptyset \\
                [DATA\_HORA]            
            \end{zed}                 


        \subsection{Esquemas estáticos}

            Esquema que define o Sensor de Umidade.
            \begin{schema}{SensorUmidade}
                umidadeLida: \nat\\
                dataHoraUmidadeLida: DATA\_HORA \\ 
                umidadeMinima: \nat \\
                umidadeIdeal: \nat \\            
            \where 
                umidadeLida \in umidadeMinima .. umidadeIdeal \\
            \end{schema}
            

            Esquema que define o Sistema referente ao funcionamento do Led de Umidade.
            \begin{schema}{SistemaLedUmidade}
                leituraUmidade: \nat \\ 
                estadoLuzUmidade: COR\_LED \\
            \where
                estadoLuzUmidade = VERDE \iff umidadeLida \in umidadeMinima .. umidadeIdeal \\
                estadoLuzUmidade = VERMELHA \iff umidadeLida \notin umidadeMinima .. umidadeIdeal \\
            \end{schema}
            
            Esquema que define o Sistema referente ao funcionamento da Irrigação.
            \begin{schema}{SistemaIrrigacao}
                leituraUmidade: \nat \\ 
                estadoIrrigador: STATUS\\
            \where
                estadoIrrigador = LIGADO \iff umidadeLida < umidadeMinima \\
                estadoIrrigador = DESLIGADO \iff umidadeLida \geq umidadeMinima \\
            \end{schema}

            \newpage
            Esquema que define o Sensor de Luminosidade.
            \begin{schema}{SensorLuminosidade}
                tempoLuzLidoDia: \nat \\
                tempoLuzIdeal: \nat \\               
            \where 
                 tempoLuzLido \leq tempoLuzIdeal \\
            \end{schema}

            Esquema que define o Sistema referente ao funcionamento do Teto Retrátil.
            \begin{schema}{SistemaTetoRetratil}
                leituraLuminosidade: \nat \\ 
                estadoTetoRetratil : TETO\_RETRATIL \\
            \where
                estadoTetoRetratil = ABERTO \iff tempoLuzLido \leq tempoLuzIdeal \\
                estadoTetoRetratil = FECHADO \iff tempoLuzLido > tempoLuzIdeal \\
            \end{schema}

            Esquema que define o agrupamento dos componentes junto ao Kit Arduíno.
            \begin{schema}{KitArduino}
                SensorUmidade \land
                SistemaLedUmidade \land 
                SistemaIrrigacao \land \\
                SensorLuminosidade \land 
                SistemaTetoRetratil \\
            \end{schema}

        \subsection{Inicialização dos Esquemas Estáticos}

		\begin{schema}{initSensorUmidade}
			\Delta SensorUmidade \\
        \where
			UmidadeLida: VAZIO\\
            dataHoraUmidadeLida: VAZIO\\
			umidadeMinima: UMIDADE\_MINIMA \\
			umidadeIdeal: UMIDADE\_IDEAL \\   
		\end{schema}

		\begin{schema}{initSistemaLedUmidade}
			\Delta SistemaLedUmidade \\
        \where
			leituraUmidade: VAZIO \\ 
			EstadoLuzUmidade: COR\_LED\_INICIAL \\
		\end{schema}

		\begin{schema}{initSistemaIrrigacao}
			\Delta SistemaIrrigacao \\
        \where
			leituraUmidade: VAZIO \\ 
			estadoIrrigador: STATUS\_INICIAL \\ 
		\end{schema}

		\begin{schema}{initSensorLuminosidade}
			\Delta SensorLuminosidade \\
            \where
			tempoLuzLidoDia: VAZIO \\ 
			tempoLuzIdeal: TEMPO\_LUZ\_IDEAL \\   
		\end{schema}

		\begin{schema}{initSistemaTetoRetratil}
			\Delta SistemaTetoRetratil \\
        \where
			leituraLuminosidade: VAZIO \\ 
			estadoTetoRetratil : TETO\_RETRATIL\_INICIAL \\ 
		\end{schema}

		\begin{schema}{initKitArduino}
			initSensorUmidade \land 
			initSistemaLedUmidade \land 
			initSistemaIrrigacao \land \\
			initSensorLuminosidade \land 
			initSistemaTetoRetratil 
		\end{schema}


        \subsection{Esquemas dinâmicos}
            
            Esquema que representa a rotina de coleta das informações do sensor de umidade.
            \begin{schema}{ColetarInformacoesSensorUmidade}
                \Delta SensorUmidade' \\
                leituraSensorUmidade? : \nat \\
                dataHoraAtual? : DATA\_HORA \\
            \where
                umidadeLida' = leituraSensorUmidade? \\
                dataHoraUmidadeLida' = dataHoraAtual? \\
            \end{schema}

            \newpage
            Esquema que representa a rotina de controle sobre a luz da umidade.
            \begin{schema}{MudarCorLedUmidade}
                \Delta SistemaLedUmidade' \\
                umidadeLida? : \nat \\
            \where
                umidadeLida \in umidadeMinima .. umidadeIdeal \implies estadoLuzUmidade' = VERDE \\
                \lor \\
                umidadeLida \notin umidadeMinima .. umidadeIdeal \implies estadoLuzUmidade' = VERMELHA \\
            \end{schema}            

            Esquema que define o Sistema referente ao funcionamento da Irrigação, quando liga, repete por a rotina 
            ColetarInformacoesSensorUmidade a cada 1 segundo, e ao atingir a umidade mínima, para de regar. Pois o delay de 1 segundo
            fará com que a umidade suba além do mínimo ficando em um nível aceitável.
            \begin{schema}{ControlarSistemaIrrigacao}
                \Delta SistemaIrrigacao' \\
                umidadeLida?: \nat \\ 
            \where
                umidadeLida < umidadeMinima \implies estadoIrrigador = LIGADO  \\
                \lor \\
                umidadeLida \geq umidadeMinima \implies estadoIrrigador = DESLIGADO  \\
            \end{schema}

            Esquema que representa a rotina de coleta das informações do sensor de luminosidade.
            \begin{schema}{ColetarInformacoesSensorLuminosidade}
                \Delta SensorLuminosidade' \\
                leituraSensorLuminosidade? : \nat \\
            \where
                tempoLuzLidoDia' = leituraSensorLuminosidade? \\
            \end{schema}            

            Esquema que representa a rotina de controle do teto retrátil.
            \begin{schema}{ControlarSistemaTetoRetratil}
                \Delta SistemaTetoRetratil' \\
                tempoLuzLido?: \nat \\             
            \where
                tempoLuzLido \leq tempoLuzIdeal\implies estadoTetoRetratil = ABERTO \\
                \lor \\
                tempoLuzLido > tempoLuzIdeal \implies estadoTetoRetratil = FECHADO \\
            \end{schema}

            \newpage
            Este esquema representa a rotina de envio de informações para o aplicativo Smart Life.
            \begin{schema}{EnviarDadosSmartLife}
                \Xi KitArduino \\
                dadosSincronizacao! : \seq \\
            \where
                dadosSincronizacao! = DadosSincronizacao\\
                POST\_API(dadosSincronizacao!)
            \end{schema}
\chapter{Análise da Ferramenta}

    A notação Z é uma linguagem importante na área de software, uma vez que oferece uma maneira especifica e formal para definir os estados e comportamentos de um sistema. Isso evita ambiguidades no desenvolvimento futuro, por se tratar de uma linguagem que utiliza uma base matemática consegue garantir uma maior confiabilidade em rotinas que fazem seu uso, garantindo assim um software de maior qualidade e com tempo de evolução reduzido.
    

% -----------------------------------------------------------------
% ELEMENTOS PÓS-TEXTUAIS
% -----------------------------------------------------------------
\postextual

\include{PosTextuais/Conclusão}

\end{document}

% -----------------------------------------------------------------
% Fim do Documento
% -----------------------------------------------------------------	