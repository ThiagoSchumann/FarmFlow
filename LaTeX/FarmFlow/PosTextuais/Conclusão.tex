\chapter{Conclusão}

Após o desenvolvimento deste projeto que consiste em criar uma especificação de requisitos formal para um sistema que fará o a coleta, análise e tomada de decisão no controle de luminosidade e umidade de uma horta inteligente. Concluindo o desenvolvimento percebeu-se que a aplicação de métodos formais é um recurso importante para garantir um software ou funcionalidade mais concisa e inequívoca, que permite ao desenvolvedor obter uma visão clara e refinada do escopo do projeto, facilitando assim a construção e manutenção de um software robusto e estável.

Outro ponto muito importante a se avaliar, é a questão da viabilidade de utilização conforme o nível de simplicidade do escopo e o nível de maturidade do time de desenvolvimento em relação à Notação Z, pois no cenário onde temos um projeto com baixa complexidade e criticidade, aliado a um time que não possui conhecimento em Notação Z, acredito não ser necessário a aplicação.

Já no cenário onde existe uma complexidade maior ou o nível de criticidade se eleva, como, por exemplo, ao lidar com  vidas humanas  ou dinheiro, se tornam imprescindíveis a aplicação de algum método formal para refinar ainda mais o escopo do projeto e assim desenvolver um software com muito mais consistência e com menos redundâncias.