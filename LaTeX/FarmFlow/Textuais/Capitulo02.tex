\chapter{Desenvolvimento}
    \section{Requisitos funcionais}
        \begin{table}[!htbp]
            \centering
            \renewcommand{\arraystretch}{1.3}
            \label{tab:quadro_requisitos}
            \begin{tabular}{| L{3cm} | L{11cm} | }
                \hline
                \textbf{RF-01} & O Software deve coletar informações referentes ao sensor de umidade do solo.\\
                \hline

                \hline
                \textbf{RF-02} & O Software deve controlar a cor de um led que representa o estado da umidade do solo.\\
                \hline                   

                \hline
                \textbf{RF-03} & O Software deve controlar a irrigação do solo.\\
                \hline

                \hline
                \textbf{RF-04} & O Software deve coletar informações referentes ao tempo de exposição à luz do solo.\\
                \hline

                \hline
                \textbf{RF-05} & o Software deve controlar o teto retrátil.\\
                \hline

                \hline
                \textbf{RF-06} & O Software deve enviar informações do solo para o Smart Life.\\
                \hline

            \end{tabular}
            \vspace{2mm}
        \end{table}

    \section{Regras de negócio}
        \begin{table}[!htbp]
            \centering
            \renewcommand{\arraystretch}{1.3}
            \label{tab:quadro_regras_negocio}
            \begin{tabular}{| L{3cm} | L{11cm} | }
                \hline
                \textbf{RN-01} & Quando a leitura de umidade for realizada e a umidade estiver igual ou acima do
                nível estipulado, nada deve ser feito.\\
                \hline

                \hline
                \textbf{RN-02} & Quando a umidade estiver abaixo do nível estipulado, inicie a irrigação e realize leituras
                a cada segundo. Pare a irrigação quando o nível máximo for atingido.\\
                \hline

                \hline
                \textbf{RN-03} & Quando a leitura de umidade for realizada e o nível estiver abaixo do nível
                configurado, apaga a luz de umidade alta.\\
                \hline

                \hline
                \textbf{RN-04} & Quando a leitura de umidade for realizada e o nível estiver acima do nível
                configurado, acende a luz de umidade alta.\\
                \hline

                \hline
                \textbf{RN-05} & Quando a leitura de luminosidade for realizada e a quantidade de luz já estiver
                insuficiente, o teto retrátil permanece aberto.\\
                \hline

                \hline
                \textbf{RN-06} & Quando a leitura de luminosidade for realizada e a quantidade de luz já estiver
                insuficiente, o teto retrátil permanece aberto.\\
                \hline

                \hline
                \textbf{RN-07} & As informações captadas pelos sensores devem ser disponibilizadas para visualização
                no aplicativo Smart Life.\\
                \hline

            \end{tabular}
            \vspace{2mm}
        \end{table}

\newpage

    \section{Especificação formal utilizando notação Z}

        \subsection{Definição de constantes e tipos}

            \begin{tabular}{|l}
                Status ::= LIGADO | DESLIGADO     \\
                CorLed :: == VERDE | VERMELHO     \\
                TetoRetratil ::= ABERTO | FECHADO
            \end{tabular}


        \subsection{Esquemas estáticos}

            \begin{schema}{IndicadorUmidade}
                leituraUmidade: \nat \\
                luzUmidade: CorLed \\
                umidadeMinima: \nat \\
                umidadeIdeal: \nat \\
            \where
                leituraUmidade \geq umidadeMinima \\
                leituraUmidade \leq umidadeIdeal \\
                luzUmidade = VERDE \iff leitura = umidadeIdeal \\
            \end{schema}

            \begin{schema}{Irrigador}
                IndicadorUmidade \\
                estadoIrrigador: Status\\
            \where
                estadoIrrigador = LIGADO \iff luzUmidade = VERMELHO \\
            \end{schema}

            \begin{schema}{KitSensorLuminosidade}
                tempoLuzLido: \nat \\
                tempoLusIdeal: \nat \\
            \where
                tempoLuzLido \leq tempoLuminosidadeIdeal \\
            \end{schema}

            \begin{schema}{ControleLuminosidadeTeto}
                KitSensorLuminosidade \\
                EstadoTetoRetratil : TetoRetratil \\
            \where
                EstadoTetoRetratil = FECHADO \iff tempoLuzLido > tempoLusIdeal \\
            \end{schema}

            \begin{schema}{KitArduino}
                SensorUmidade \\
                ReleIrrigacao \\
                KitSensorLuminosidade \\
                ReleTetoRetratil \\
                dadosSincronizacao : \seq Caracter \\
            \end{schema}

        \subsection{Inicialização dos Esquemas Estáticos}

            \begin{schema}{InitIndicadorUmidade}
                \Delta IndicadorUmidade \\
            \where
                umidadeMinima = 60 \\
                umidadeIdeal = 65 \\
                luzUmidade = VERDE \\
            \end{schema}

            \begin{schema}{InitIrrigador}
                \Delta Irrigador \\
            \where
                estadoIrrigador = DESLIGADO \\
            \end{schema}

            \begin{schema}{InitKitSensorLuminosidade}
                \Delta KitSensorLuminosidade \\
            \where
                tempoLusIdeal = 6 \\
            \end{schema}

            \begin{schema}{InitControleLuminosidadeTeto}
                \Delta ControleLuminosidadeTeto \\
            \where
                EstadoTetoRetratil = ABERTO \\
            \end{schema}

            \begin{schema}{InitKitArduino}
                \Delta KitArduino \\
            \where
                leituraUmidade = 0\\
                tempoLuzLido  = 0\\
                dadosSincronizacao = \emptyset \\
            \end{schema}

        \subsection{Esquemas dinâmicos}
            ---------------------------------------------------------------------------------------------------------------------
            Este é o modelo para criação de esquemas dinâmicos e deve ser copiado e replicado ao criar os seus modelos
            \begin{schema}{Model}
                \Delta model' \\
                edit ?: Status \\
                luzUmidade ?: CorLed \\
                read !: \nat \\                
            \where
                \\
            \end{schema}
            ---------------------------------------------------------------------------------------------------------------------
                - ColetarInformacoesUmidade
                - IniciaIrrigacao
                - ParaIrrigacao
                - AcenderLedVermelho
                - AcenderLedVerde
                - ColetarInformacoesTempoLuz
                - AbrirTetoRetratil
                - FecharTetoRetratil
                - 
                - 
                - 
            ---------------------------------------------------------------------------------------------------------------------            


            Este esquema representa a rotina de alterar o estado do irrigador e da luz de umidade conforme a leitura do sensor de umidade.

            \begin{schema}{Irrigar}
                \Delta KitArduino' \\
                estadoIrrigador ?: Status \\
                luzUmidade ?: CorLed \\
                leituraUmidade !: \nat \\
                umidadeMinima !: \nat \\
                umidadeIdeal !: \nat \\
            \where
                estadoIrrigador' = LIGADO \iff leituraUmidade \leq umidadeMinima \\
                estadoIrrigador' = DESLIGADO \iff leituraUmidade \geq umidadeIdeal \\
                luzUmidade' = VERDE \iff leituraUmidade \geq umidadeIdeal \\
                luzUmidade' = VERMELHO \iff leituraUmidade \leq umidadeIdeal \\
            \end{schema}


            Este esquema representa a rotina de controle de abertura/fechamento do teto retratil da horta, permitindo que a horta receba apenas a quantidade de luz ideal para seu desenvolvimento.

            \begin{schema}{ControlarTetoRetratil}
                \Delta KitArduino' \\
                EstadoTetoRetratil ?: TetoRetratil \\
                tempoLuzLido !: \nat \\
                tempoLusIdeal !: \nat \\
            \where
                EstadoTetoRetratil' = ABERTO \iff tempoLuzLido \leq tempoLusIdeal \\
                EstadoTetoRetratil' = FECHADO \iff tempoLuzLido > tempoLusIdeal \\
            \end{schema}


            Este esquema representa a rotina de envio de informações para o aplicativo Smart Life.

            \begin{schema}{EnviarDadosSmartLife}
                KitArduino \\
                dadosSincronizacao !: \seq Caracter \\
            \where
                EnviarDadosAPI(dadosSincronizacao) \\
            \end{schema}

