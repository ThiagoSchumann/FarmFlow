\chapter{Desenvolvimento}
    \section{Requisitos funcionais}
        \begin{table}[!htbp]
            \centering
            \renewcommand{\arraystretch}{1.3}
            \label{tab:quadro_requisitos}
            \begin{tabular}{| L{3cm} | L{11cm} | }
                \hline
                \textbf{RF-01} & O Software deve coletar informações referentes ao sensor de umidade do solo.\\
                \hline

                \hline
                \textbf{RF-02} & O Software deve controlar a cor de um led que representa o estado da umidade do solo.\\
                \hline                   

                \hline
                \textbf{RF-03} & O Software deve controlar a irrigação do solo.\\
                \hline

                \hline
                \textbf{RF-04} & O Software deve coletar informações referentes ao tempo de exposição à luz do solo.\\
                \hline

                \hline
                \textbf{RF-05} & O Software deve controlar o teto retrátil.\\
                \hline

                \hline
                \textbf{RF-06} & O Software deve enviar informações do solo para o Smart Life.\\
                \hline

            \end{tabular}
            \vspace{2mm}
        \end{table}

    \section{Regras de negócio}
        \begin{table}[!htbp]
            \centering
            \renewcommand{\arraystretch}{1.3}
            \label{tab:quadro_regras_negocio}
            \begin{tabular}{| L{3cm} | L{11cm} | }
                \hline
                \textbf{RN-01} & A leitura de umidade referente ao sensor deve realizar leituras a cada 45 minutos.\\
                \hline

                \hline
                \textbf{RN-02} & Quando a leitura de umidade for realizada e a umidade estiver dentro dos limites, luz de umidade fica verde, se não, luz de umidade fica vermelha.\\
                \hline                

                \hline
                \textbf{RN-03} & Quando a leitura da umidade estiver abaixo do nível estipulado, inicie a irrigação, realize leituras a cada segundo e pare a irrigação quando o nível mínimo for atingido. Pois o delay de 1 segundo fará com que o nível de umidade fique acima do nível mínimo.\\
                \hline

                \hline
                \textbf{RN-04} & A leitura de tempo de exposição a luz referente ao sensor deve realizar leituras a cada 45 minutos.\\
                \hline

                \hline
                \textbf{RN-05} & Quando a leitura de luminosidade for realizada e a quantidade de luz já estiver
                insuficiente, o teto retrátil permanece aberto, se não, ele é fechado.\\
                \hline

                \hline
                \textbf{RN-06} & As informações captadas pelos sensores devem ser disponibilizadas para visualização
                no aplicativo Smart Life.\\
                \hline

            \end{tabular}
            \vspace{2mm}
        \end{table}

\newpage

    \section{Especificação formal utilizando notação Z}

        \subsection{Definição de constantes e tipos}

            \begin{zed}
                STATUS ::= LIGADO | DESLIGADO     \\
                COR\_LED ::= VERDE | VERMELHO     \\
                TETO\_RETRATIL ::= ABERTO | FECHADO \\    
                UMIDADE\_MINIMA ::= 60 \\
                UMIDADE\_IDEAL ::= 75 \\
                COR\_LED\_INICIAL :: VERDE \\
                STATUS\_INICIAL :: DESLIGADO \\
                TEMPO\_LUZ\_IDEAL ::= 6 \\
                TETO\_RETRATIL\_INICIAL ::= ABERTO \\
                VAZIO :: = \emptyset \\
                [DATA\_HORA]            
            \end{zed}                 


        \subsection{Esquemas estáticos}

            Esquema que define o Sensor de Umidade.
            \begin{schema}{SensorUmidade}
                umidadeLida: \nat\\
                dataHoraUmidadeLida: DATA\_HORA \\ 
                umidadeMinima: \nat \\
                umidadeIdeal: \nat \\            
            \where 
                umidadeLida \in umidadeMinima .. umidadeIdeal \\
            \end{schema}
            

            Esquema que define o Sistema referente ao funcionamento do Led de Umidade.
            \begin{schema}{SistemaLedUmidade}
                leituraUmidade: \nat \\ 
                estadoLuzUmidade: COR\_LED \\
            \where
                estadoLuzUmidade = VERDE \iff umidadeLida \in umidadeMinima .. umidadeIdeal \\
                estadoLuzUmidade = VERMELHA \iff umidadeLida \notin umidadeMinima .. umidadeIdeal \\
            \end{schema}
            
            Esquema que define o Sistema referente ao funcionamento da Irrigação.
            \begin{schema}{SistemaIrrigacao}
                leituraUmidade: \nat \\ 
                estadoIrrigador: STATUS\\
            \where
                estadoIrrigador = LIGADO \iff umidadeLida < umidadeMinima \\
                estadoIrrigador = DESLIGADO \iff umidadeLida \geq umidadeMinima \\
            \end{schema}

            \newpage
            Esquema que define o Sensor de Luminosidade.
            \begin{schema}{SensorLuminosidade}
                tempoLuzLidoDia: \nat \\
                tempoLuzIdeal: \nat \\               
            \where 
                 tempoLuzLido \leq tempoLuzIdeal \\
            \end{schema}

            Esquema que define o Sistema referente ao funcionamento do Teto Retrátil.
            \begin{schema}{SistemaTetoRetratil}
                leituraLuminosidade: \nat \\ 
                estadoTetoRetratil : TETO\_RETRATIL \\
            \where
                estadoTetoRetratil = ABERTO \iff tempoLuzLido \leq tempoLuzIdeal \\
                estadoTetoRetratil = FECHADO \iff tempoLuzLido > tempoLuzIdeal \\
            \end{schema}

            Esquema que define o agrupamento dos componentes junto ao Kit Arduíno.
            \begin{schema}{KitArduino}
                SensorUmidade \land
                SistemaLedUmidade \land 
                SistemaIrrigacao \land \\
                SensorLuminosidade \land 
                SistemaTetoRetratil \\
            \end{schema}

        \subsection{Inicialização dos Esquemas Estáticos}

		\begin{schema}{initSensorUmidade}
			\Delta SensorUmidade \\
        \where
			UmidadeLida: VAZIO\\
            dataHoraUmidadeLida: VAZIO\\
			umidadeMinima: UMIDADE\_MINIMA \\
			umidadeIdeal: UMIDADE\_IDEAL \\   
		\end{schema}

		\begin{schema}{initSistemaLedUmidade}
			\Delta SistemaLedUmidade \\
        \where
			leituraUmidade: VAZIO \\ 
			EstadoLuzUmidade: COR\_LED\_INICIAL \\
		\end{schema}

		\begin{schema}{initSistemaIrrigacao}
			\Delta SistemaIrrigacao \\
        \where
			leituraUmidade: VAZIO \\ 
			estadoIrrigador: STATUS\_INICIAL \\ 
		\end{schema}

		\begin{schema}{initSensorLuminosidade}
			\Delta SensorLuminosidade \\
            \where
			tempoLuzLidoDia: VAZIO \\ 
			tempoLuzIdeal: TEMPO\_LUZ\_IDEAL \\   
		\end{schema}

		\begin{schema}{initSistemaTetoRetratil}
			\Delta SistemaTetoRetratil \\
        \where
			leituraLuminosidade: VAZIO \\ 
			estadoTetoRetratil : TETO\_RETRATIL\_INICIAL \\ 
		\end{schema}

		\begin{schema}{initKitArduino}
			initSensorUmidade \land 
			initSistemaLedUmidade \land 
			initSistemaIrrigacao \land \\
			initSensorLuminosidade \land 
			initSistemaTetoRetratil 
		\end{schema}


        \subsection{Esquemas dinâmicos}
            
            Esquema que representa a rotina de coleta das informações do sensor de umidade.
            \begin{schema}{ColetarInformacoesSensorUmidade}
                \Delta SensorUmidade' \\
                leituraSensorUmidade? : \nat \\
                dataHoraAtual? : DATA\_HORA \\
            \where
                umidadeLida' = leituraSensorUmidade? \\
                dataHoraUmidadeLida' = dataHoraAtual? \\
            \end{schema}

            \newpage
            Esquema que representa a rotina de controle sobre a luz da umidade.
            \begin{schema}{MudarCorLedUmidade}
                \Delta SistemaLedUmidade' \\
                umidadeLida? : \nat \\
            \where
                umidadeLida \in umidadeMinima .. umidadeIdeal \implies estadoLuzUmidade' = VERDE \\
                \lor \\
                umidadeLida \notin umidadeMinima .. umidadeIdeal \implies estadoLuzUmidade' = VERMELHA \\
            \end{schema}            

            Esquema que define o Sistema referente ao funcionamento da Irrigação, quando liga, repete por a rotina 
            ColetarInformacoesSensorUmidade a cada 1 segundo, e ao atingir a umidade mínima, para de regar. Pois o delay de 1 segundo
            fará com que a umidade suba além do mínimo ficando em um nível aceitável.
            \begin{schema}{ControlarSistemaIrrigacao}
                \Delta SistemaIrrigacao' \\
                umidadeLida?: \nat \\ 
            \where
                umidadeLida < umidadeMinima \implies estadoIrrigador = LIGADO  \\
                \lor \\
                umidadeLida \geq umidadeMinima \implies estadoIrrigador = DESLIGADO  \\
            \end{schema}

            Esquema que representa a rotina de coleta das informações do sensor de luminosidade.
            \begin{schema}{ColetarInformacoesSensorLuminosidade}
                \Delta SensorLuminosidade' \\
                leituraSensorLuminosidade? : \nat \\
            \where
                tempoLuzLidoDia' = leituraSensorLuminosidade? \\
            \end{schema}            

            Esquema que representa a rotina de controle do teto retrátil.
            \begin{schema}{ControlarSistemaTetoRetratil}
                \Delta SistemaTetoRetratil' \\
                tempoLuzLido?: \nat \\             
            \where
                tempoLuzLido \leq tempoLuzIdeal\implies estadoTetoRetratil = ABERTO \\
                \lor \\
                tempoLuzLido > tempoLuzIdeal \implies estadoTetoRetratil = FECHADO \\
            \end{schema}

            \newpage
            Este esquema representa a rotina de envio de informações para o aplicativo Smart Life.
            \begin{schema}{EnviarDadosSmartLife}
                \Xi KitArduino \\
                dadosSincronizacao! : \seq \\
            \where
                dadosSincronizacao! = DadosSincronizacao\\
                POST\_API(dadosSincronizacao!)
            \end{schema}